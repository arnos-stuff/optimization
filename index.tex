% Options for packages loaded elsewhere
\PassOptionsToPackage{unicode}{hyperref}
\PassOptionsToPackage{hyphens}{url}
\PassOptionsToPackage{dvipsnames,svgnames,x11names}{xcolor}
%
\documentclass[
  letterpaper,
  DIV=11,
  numbers=noendperiod]{scrreprt}

\usepackage{amsmath,amssymb}
\usepackage{lmodern}
\usepackage{iftex}
\ifPDFTeX
  \usepackage[T1]{fontenc}
  \usepackage[utf8]{inputenc}
  \usepackage{textcomp} % provide euro and other symbols
\else % if luatex or xetex
  \usepackage{unicode-math}
  \defaultfontfeatures{Scale=MatchLowercase}
  \defaultfontfeatures[\rmfamily]{Ligatures=TeX,Scale=1}
\fi
% Use upquote if available, for straight quotes in verbatim environments
\IfFileExists{upquote.sty}{\usepackage{upquote}}{}
\IfFileExists{microtype.sty}{% use microtype if available
  \usepackage[]{microtype}
  \UseMicrotypeSet[protrusion]{basicmath} % disable protrusion for tt fonts
}{}
\makeatletter
\@ifundefined{KOMAClassName}{% if non-KOMA class
  \IfFileExists{parskip.sty}{%
    \usepackage{parskip}
  }{% else
    \setlength{\parindent}{0pt}
    \setlength{\parskip}{6pt plus 2pt minus 1pt}}
}{% if KOMA class
  \KOMAoptions{parskip=half}}
\makeatother
\usepackage{xcolor}
\setlength{\emergencystretch}{3em} % prevent overfull lines
\setcounter{secnumdepth}{5}
% Make \paragraph and \subparagraph free-standing
\ifx\paragraph\undefined\else
  \let\oldparagraph\paragraph
  \renewcommand{\paragraph}[1]{\oldparagraph{#1}\mbox{}}
\fi
\ifx\subparagraph\undefined\else
  \let\oldsubparagraph\subparagraph
  \renewcommand{\subparagraph}[1]{\oldsubparagraph{#1}\mbox{}}
\fi

\usepackage{color}
\usepackage{fancyvrb}
\newcommand{\VerbBar}{|}
\newcommand{\VERB}{\Verb[commandchars=\\\{\}]}
\DefineVerbatimEnvironment{Highlighting}{Verbatim}{commandchars=\\\{\}}
% Add ',fontsize=\small' for more characters per line
\usepackage{framed}
\definecolor{shadecolor}{RGB}{241,243,245}
\newenvironment{Shaded}{\begin{snugshade}}{\end{snugshade}}
\newcommand{\AlertTok}[1]{\textcolor[rgb]{0.68,0.00,0.00}{#1}}
\newcommand{\AnnotationTok}[1]{\textcolor[rgb]{0.37,0.37,0.37}{#1}}
\newcommand{\AttributeTok}[1]{\textcolor[rgb]{0.40,0.45,0.13}{#1}}
\newcommand{\BaseNTok}[1]{\textcolor[rgb]{0.68,0.00,0.00}{#1}}
\newcommand{\BuiltInTok}[1]{\textcolor[rgb]{0.00,0.23,0.31}{#1}}
\newcommand{\CharTok}[1]{\textcolor[rgb]{0.13,0.47,0.30}{#1}}
\newcommand{\CommentTok}[1]{\textcolor[rgb]{0.37,0.37,0.37}{#1}}
\newcommand{\CommentVarTok}[1]{\textcolor[rgb]{0.37,0.37,0.37}{\textit{#1}}}
\newcommand{\ConstantTok}[1]{\textcolor[rgb]{0.56,0.35,0.01}{#1}}
\newcommand{\ControlFlowTok}[1]{\textcolor[rgb]{0.00,0.23,0.31}{#1}}
\newcommand{\DataTypeTok}[1]{\textcolor[rgb]{0.68,0.00,0.00}{#1}}
\newcommand{\DecValTok}[1]{\textcolor[rgb]{0.68,0.00,0.00}{#1}}
\newcommand{\DocumentationTok}[1]{\textcolor[rgb]{0.37,0.37,0.37}{\textit{#1}}}
\newcommand{\ErrorTok}[1]{\textcolor[rgb]{0.68,0.00,0.00}{#1}}
\newcommand{\ExtensionTok}[1]{\textcolor[rgb]{0.00,0.23,0.31}{#1}}
\newcommand{\FloatTok}[1]{\textcolor[rgb]{0.68,0.00,0.00}{#1}}
\newcommand{\FunctionTok}[1]{\textcolor[rgb]{0.28,0.35,0.67}{#1}}
\newcommand{\ImportTok}[1]{\textcolor[rgb]{0.00,0.46,0.62}{#1}}
\newcommand{\InformationTok}[1]{\textcolor[rgb]{0.37,0.37,0.37}{#1}}
\newcommand{\KeywordTok}[1]{\textcolor[rgb]{0.00,0.23,0.31}{#1}}
\newcommand{\NormalTok}[1]{\textcolor[rgb]{0.00,0.23,0.31}{#1}}
\newcommand{\OperatorTok}[1]{\textcolor[rgb]{0.37,0.37,0.37}{#1}}
\newcommand{\OtherTok}[1]{\textcolor[rgb]{0.00,0.23,0.31}{#1}}
\newcommand{\PreprocessorTok}[1]{\textcolor[rgb]{0.68,0.00,0.00}{#1}}
\newcommand{\RegionMarkerTok}[1]{\textcolor[rgb]{0.00,0.23,0.31}{#1}}
\newcommand{\SpecialCharTok}[1]{\textcolor[rgb]{0.37,0.37,0.37}{#1}}
\newcommand{\SpecialStringTok}[1]{\textcolor[rgb]{0.13,0.47,0.30}{#1}}
\newcommand{\StringTok}[1]{\textcolor[rgb]{0.13,0.47,0.30}{#1}}
\newcommand{\VariableTok}[1]{\textcolor[rgb]{0.07,0.07,0.07}{#1}}
\newcommand{\VerbatimStringTok}[1]{\textcolor[rgb]{0.13,0.47,0.30}{#1}}
\newcommand{\WarningTok}[1]{\textcolor[rgb]{0.37,0.37,0.37}{\textit{#1}}}

\providecommand{\tightlist}{%
  \setlength{\itemsep}{0pt}\setlength{\parskip}{0pt}}\usepackage{longtable,booktabs,array}
\usepackage{calc} % for calculating minipage widths
% Correct order of tables after \paragraph or \subparagraph
\usepackage{etoolbox}
\makeatletter
\patchcmd\longtable{\par}{\if@noskipsec\mbox{}\fi\par}{}{}
\makeatother
% Allow footnotes in longtable head/foot
\IfFileExists{footnotehyper.sty}{\usepackage{footnotehyper}}{\usepackage{footnote}}
\makesavenoteenv{longtable}
\usepackage{graphicx}
\makeatletter
\def\maxwidth{\ifdim\Gin@nat@width>\linewidth\linewidth\else\Gin@nat@width\fi}
\def\maxheight{\ifdim\Gin@nat@height>\textheight\textheight\else\Gin@nat@height\fi}
\makeatother
% Scale images if necessary, so that they will not overflow the page
% margins by default, and it is still possible to overwrite the defaults
% using explicit options in \includegraphics[width, height, ...]{}
\setkeys{Gin}{width=\maxwidth,height=\maxheight,keepaspectratio}
% Set default figure placement to htbp
\makeatletter
\def\fps@figure{htbp}
\makeatother
\newlength{\cslhangindent}
\setlength{\cslhangindent}{1.5em}
\newlength{\csllabelwidth}
\setlength{\csllabelwidth}{3em}
\newlength{\cslentryspacingunit} % times entry-spacing
\setlength{\cslentryspacingunit}{\parskip}
\newenvironment{CSLReferences}[2] % #1 hanging-ident, #2 entry spacing
 {% don't indent paragraphs
  \setlength{\parindent}{0pt}
  % turn on hanging indent if param 1 is 1
  \ifodd #1
  \let\oldpar\par
  \def\par{\hangindent=\cslhangindent\oldpar}
  \fi
  % set entry spacing
  \setlength{\parskip}{#2\cslentryspacingunit}
 }%
 {}
\usepackage{calc}
\newcommand{\CSLBlock}[1]{#1\hfill\break}
\newcommand{\CSLLeftMargin}[1]{\parbox[t]{\csllabelwidth}{#1}}
\newcommand{\CSLRightInline}[1]{\parbox[t]{\linewidth - \csllabelwidth}{#1}\break}
\newcommand{\CSLIndent}[1]{\hspace{\cslhangindent}#1}

\KOMAoption{captions}{tableheading}
\makeatletter
\@ifpackageloaded{tcolorbox}{}{\usepackage[many]{tcolorbox}}
\@ifpackageloaded{fontawesome5}{}{\usepackage{fontawesome5}}
\definecolor{quarto-callout-color}{HTML}{909090}
\definecolor{quarto-callout-note-color}{HTML}{0758E5}
\definecolor{quarto-callout-important-color}{HTML}{CC1914}
\definecolor{quarto-callout-warning-color}{HTML}{EB9113}
\definecolor{quarto-callout-tip-color}{HTML}{00A047}
\definecolor{quarto-callout-caution-color}{HTML}{FC5300}
\definecolor{quarto-callout-color-frame}{HTML}{acacac}
\definecolor{quarto-callout-note-color-frame}{HTML}{4582ec}
\definecolor{quarto-callout-important-color-frame}{HTML}{d9534f}
\definecolor{quarto-callout-warning-color-frame}{HTML}{f0ad4e}
\definecolor{quarto-callout-tip-color-frame}{HTML}{02b875}
\definecolor{quarto-callout-caution-color-frame}{HTML}{fd7e14}
\makeatother
\makeatletter
\makeatother
\makeatletter
\@ifpackageloaded{bookmark}{}{\usepackage{bookmark}}
\makeatother
\makeatletter
\@ifpackageloaded{caption}{}{\usepackage{caption}}
\AtBeginDocument{%
\ifdefined\contentsname
  \renewcommand*\contentsname{Table of contents}
\else
  \newcommand\contentsname{Table of contents}
\fi
\ifdefined\listfigurename
  \renewcommand*\listfigurename{List of Figures}
\else
  \newcommand\listfigurename{List of Figures}
\fi
\ifdefined\listtablename
  \renewcommand*\listtablename{List of Tables}
\else
  \newcommand\listtablename{List of Tables}
\fi
\ifdefined\figurename
  \renewcommand*\figurename{Figure}
\else
  \newcommand\figurename{Figure}
\fi
\ifdefined\tablename
  \renewcommand*\tablename{Table}
\else
  \newcommand\tablename{Table}
\fi
}
\@ifpackageloaded{float}{}{\usepackage{float}}
\floatstyle{ruled}
\@ifundefined{c@chapter}{\newfloat{codelisting}{h}{lop}}{\newfloat{codelisting}{h}{lop}[chapter]}
\floatname{codelisting}{Listing}
\newcommand*\listoflistings{\listof{codelisting}{List of Listings}}
\makeatother
\makeatletter
\@ifpackageloaded{caption}{}{\usepackage{caption}}
\@ifpackageloaded{subcaption}{}{\usepackage{subcaption}}
\makeatother
\makeatletter
\@ifpackageloaded{tcolorbox}{}{\usepackage[many]{tcolorbox}}
\makeatother
\makeatletter
\@ifundefined{shadecolor}{\definecolor{shadecolor}{rgb}{.97, .97, .97}}
\makeatother
\makeatletter
\makeatother
\ifLuaTeX
  \usepackage{selnolig}  % disable illegal ligatures
\fi
\IfFileExists{bookmark.sty}{\usepackage{bookmark}}{\usepackage{hyperref}}
\IfFileExists{xurl.sty}{\usepackage{xurl}}{} % add URL line breaks if available
\urlstyle{same} % disable monospaced font for URLs
\hypersetup{
  pdftitle={Optimization},
  pdfauthor={Arno Veletanlic},
  colorlinks=true,
  linkcolor={blue},
  filecolor={Maroon},
  citecolor={Blue},
  urlcolor={Blue},
  pdfcreator={LaTeX via pandoc}}

\title{Optimization}
\author{Arno Veletanlic}
\date{3/22/23}

\begin{document}
\maketitle
\ifdefined\Shaded\renewenvironment{Shaded}{\begin{tcolorbox}[enhanced, borderline west={3pt}{0pt}{shadecolor}, breakable, frame hidden, sharp corners, interior hidden, boxrule=0pt]}{\end{tcolorbox}}\fi

\renewcommand*\contentsname{Table of contents}
{
\hypersetup{linkcolor=}
\setcounter{tocdepth}{2}
\tableofcontents
}
\bookmarksetup{startatroot}

\hypertarget{preface}{%
\chapter*{Preface}\label{preface}}
\addcontentsline{toc}{chapter}{Preface}

\markboth{Preface}{Preface}

This is a Quarto book.

To learn more about Quarto books visit
\url{https://quarto.org/docs/books}.

\bookmarksetup{startatroot}

\hypertarget{review-of-optimization-basics}{%
\chapter{Review of Optimization
basics}\label{review-of-optimization-basics}}

\hypertarget{first-definitions}{%
\section{First definitions}\label{first-definitions}}

\hypertarget{why-we-need-calculus}{%
\subsection{Why we need Calculus}\label{why-we-need-calculus}}

Suppose you have a function \(f(x)\) and you want to find the value of
\(x\) that makes \(f(x)\) as large as possible. This is called a
\emph{maximum}.

Our issue is thus to find various forms of maxima and minima. We denote
this problem as follows:

\begin{longtable}[]{@{}
  >{\raggedright\arraybackslash}p{(\columnwidth - 6\tabcolsep) * \real{0.1765}}
  >{\raggedright\arraybackslash}p{(\columnwidth - 6\tabcolsep) * \real{0.1765}}
  >{\raggedright\arraybackslash}p{(\columnwidth - 6\tabcolsep) * \real{0.2941}}
  >{\raggedright\arraybackslash}p{(\columnwidth - 6\tabcolsep) * \real{0.3529}}@{}}
\toprule()
\begin{minipage}[b]{\linewidth}\raggedright
Number
\end{minipage} & \begin{minipage}[b]{\linewidth}\raggedright
Objective Name
\end{minipage} & \begin{minipage}[b]{\linewidth}\raggedright
Mathematical expression
\end{minipage} & \begin{minipage}[b]{\linewidth}\raggedright
Key elements
\end{minipage} \\
\midrule()
\endhead
\((1)\) & Unconstrained Maximization & \(\max_x f(x)\) & The only
restriction on \(f\) is the nature of its admissible inputs
\(\mathbf{dom}(f)\) \\
\((2)\) & Finding the global maximizer &
\(x^\star \gets \argmax_x f(x)\) & There is generally a more restriction
set of assumptions than in \((1)\) to find the precise \(x^\star\) for
which \(f\) is minimized \\
\((3)\) & Finding the a maximiser within a constraint set &
\(\underset{x\in\mathcal{X}}{\max f(x)}\) & In this case, there is no
possibility for \(x\) to go outside of a predefined set \(\mathcal{X}\),
this is very useful in real-life applications \\
\bottomrule()
\end{longtable}

When is calculus useful and not useful ? Here's a first reason it might
or might not be useful: the \textbf{domain} \(\mathcal{X}\) of the input
variable, or (which is often the same), the domain \(\mathbf{dom}(f)\)
or our function.

A pretty important part of the problem is what we call the \emph{domain}
of the function. This is the set of values that \(x\) can take on. For
example, if \(f(x) = x^2\), then the domain is the set of real numbers
\(\mathbb{R}\).

Domains change everything because, for example, if your domain
\(\mathbf{dom}(X)\) is discrete, then the image \(\mathbf{im}_f(X)\)
(also denoted \(f(\mathbf{dom}(X))\)) is also discrete, and you just end
up with the array-sorting problem,
\href{https://en.wikipedia.org/wiki/Quicksort}{for which efficient
algorithms like quicksort exist.}.

\textbf{The derivative} of a function
\(f(x)\colon \mathbb{R}\to\mathbb{R}\) is a function \(f'(x)\) is just
what everyone knows, that is the limit of

\[f'(x) = \lim_{h\to 0} \frac{f(x+h) - f(x)}{h}\]

When you have a function of several variables, say
\(f(x_1, x_2, \dots, x_n)\), whose values are still real numbers, then
you can still behave as if nothing happened, because if you fix the
\(n-1\) variables

\[\bar{x}_{\scriptscriptstyle-i} \coloneqq (\bar{x}_1,\ldots,\bar{x}_{\scriptscriptstyle i-1},x_{\scriptscriptstyle i+1},\dots\bar{x}_n)\]

Then the \textbf{univariate function} \(g(x_i)\) is still a function of
a single variable, and you can still take its derivative:

\[ g\colon x \mapsto f(\bar{x}_{\scriptscriptstyle-i}, x) \coloneqq f(\bar{x}_1,\ldots,\bar{x}_{\scriptscriptstyle i-1},x_i, x_{\scriptscriptstyle i+1},\dots\bar{x}_n)\]

We have a bunch of fancy notation for this, namely
\(\partial_i f(\bar{x})\) for short, or
\(\frac{\partial f}{\partial x_i}(\bar{x})\). We also sometimes speak of
the \textbf{directional derivative} of \(f\) in the direction of
\(\bar{x}\), which is often denoted \(D_{\bar{x}}f\).

What is not clear though, is how we relate each derivative to the
others. How does \(\partial_i f(\bar{x})\) relate to
\(\partial_j f(\bar{x})\)?

There is a good theorem to explain this !

\hypertarget{the-jacobian}{%
\subsection{The Jacobian}\label{the-jacobian}}

The theorem that defines and cements the derivatives is the following:

\begin{quote}
Let \(f\colon \mathbb{R}^n \to \mathbb{R}^m\) be a function. Then, at
each point \(x\) where \(f\) is differentiable, there is a
\textbf{unique matrix} \(M(x)\) so that:
\[ f(x+h) \approx f(x) + M(x)h +\varepsilon (h)\] where
\(\varepsilon (h)\) is a small vector in \(\mathbb{R}^m\) that goes to
\(0\) faster than \(h\) does\footnote{in the sense of a norm:
  \(\lvert x\rvert\), \(\varepsilon(h)/\lvert x\rvert\to 0\) as
  \(\lvert x\rvert\to 0\).}.
\end{quote}

\hypertarget{the-proof}{%
\subsection{The Proof}\label{the-proof}}

The uniqueness is fairly easy to prove, because if there are two
matrices \(M_1\) and \(M_2\) such that \(f(x+h) \approx f(x) + M_1h\)
and \(f(x+h) \approx f(x) + M_2h\), then we can subtract the two
equations, \[(M_1 - M_2)h = \varepsilon_1 (h) + \varepsilon_2 (h)\]

divide by \(\lvert{h}\rvert\) to get that the LHS tends to \(M_1 - M_2\)
as \(h\to 0\), and the RHS tends to \(0\) as \(h\to 0\), so we get that
\(M_1 - M_2 = 0\).

The existence is essentially the notion that we take matrices as
rectangular arrays of numbers. In that sense, if the matrix \(M(x)\) has
coefficients \(\left(\smash{\alpha_{ij}(x)}\right)_{i,j}\), then we can
write, our equation is merely a statement that is 1-D (which we know how
to deal with):

\begin{aligned}
f(x+h) &= f(x) + M(x)h +\varepsilon (h) &\\
\iff f_j(x+h) &= f_j(x) + \sum_{i=1}^m \alpha_{ij}(x)h_j + \varepsilon_j (h) &\forall j\in\{1,\dots m\} \\
\iff f_j(x_i+h_i) &= f_j(x_i) + \sum_{i=1}^m \alpha_{ij}(x_i)h_j + \varepsilon_j (h_i) &\forall j\in\{1,\dots m\},\,\forall i\in\{1,\dots n\} \\
\end{aligned}

This last set of equation is just a \(m\times n\) grid of first order
approximations for each \(f_j\) at each \(x_i\), and we can solve for
the \(\alpha_{ij}\) by just taking the limit as \(h_i\to 0\) for each
\(i\)\footnote{There is a subtelty here, because we can't really just
  let any \(h_i\) to zero before the others arbitrarily, but if we do it
  in any order according to most used norms (here we'll use the
  euclidean norm \(\lvert \cdot \rvert\)), then we get that the limit
  exists and is unique.)}.

If you hadn't guessed before, each of these equations are looking very
much like
\[\varphi(x+h) \approx \varphi(x) + \varphi'(x)h + \varepsilon(h)\]

we know in 1D, and we can just take the limit as \(h\to 0\) to get that
\(\varphi'(x) = \alpha_{ij}(x)\).

What this proves is 3 things:

\begin{enumerate}
\def\labelenumi{\arabic{enumi}.}
\tightlist
\item
  The Jacobian is a matrix, and it is unique.
\item
  The Jacobian is a linear operator, and it is continuous.
\item
  The Jacobian matrix is populated by the partial derivatives of \(f\),
  that is
\end{enumerate}

\[ \textbf{Jac}[f](x) = \left[\dfrac{\partial f_j}{\partial x_i}(x)\right]_{\substack{1\leq j \leq m\\1\leq j \leq n}}\]

\begin{tcolorbox}[enhanced jigsaw, toptitle=1mm, colbacktitle=quarto-callout-caution-color!10!white, toprule=.15mm, opacitybacktitle=0.6, opacityback=0, left=2mm, titlerule=0mm, breakable, coltitle=black, arc=.35mm, bottomrule=.15mm, bottomtitle=1mm, leftrule=.75mm, rightrule=.15mm, colframe=quarto-callout-caution-color-frame, title=\textcolor{quarto-callout-caution-color}{\faFire}\hspace{0.5em}{Careful about the dimensions !}, colback=white]

The jacobian sends points from a \(n\)-dimensional space to a
\(m\)-dimensional space, so the Jacobian matrix is a \(m\times n\)
matrix. This is why the partial derivatives are written in the order
\(\dfrac{\partial f_j}{\partial x_i}\), because the \(j\)th row of the
matrix is the partial derivatives of \(f_j\) with respect to the
\(x_i\)'s.

This is very counterintuitive to anyone who works with the gradient
\(\nabla f(x)\), which is a \(n\)-dimensional vector, and is the
transpose of the Jacobian matrix.

\end{tcolorbox}

\hypertarget{a-small-example-using-python}{%
\subsection{A small example using
python}\label{a-small-example-using-python}}

Let's take a look at a small example I made using python's
\texttt{sympy} library. Through \href{https://www.math3d.org/}{graphing
increasingly complex 2D functions on an online tool}, I found a function
that is complex enough for us to have fun, but not too complex that we
can't understand it.

The function is:

\[G(x,y) = \log\left[1+\left(x^{\frac{1}{3}} - y^{\frac{2}{3}}\right)^2\left(0.1+\sin \left(x\right)^2\cdot \cos \left(x\right)^2\right)^{-1}\right]
\]

But I think it's easier to understand if we write it as:

\[ G(x,y) = \log\left(1 + \dfrac{A(x,y)^2}{\gamma + B(x,y)^2}\right)\]

with \(A(x,y) = x^{\frac{1}{3}} - y^{\frac{2}{3}}\) and
\(B(x,y) = \sin \left(x\right)\cdot \cos \left(x\right)\).

We can plot this function using \texttt{plotly}:

\begin{Shaded}
\begin{Highlighting}[]
\ImportTok{import}\NormalTok{ plotly.graph\_objects }\ImportTok{as}\NormalTok{ go}
\ImportTok{import}\NormalTok{ requests }\ImportTok{as}\NormalTok{ rq}
\ImportTok{import}\NormalTok{ json}
\ImportTok{from}\NormalTok{ pathlib }\ImportTok{import}\NormalTok{ Path}

\NormalTok{template }\OperatorTok{=}\NormalTok{ json.loads(}
\NormalTok{    Path(}\StringTok{"../plotlyTemplate.json"}\NormalTok{).read\_text()}
\NormalTok{)}



\NormalTok{data }\OperatorTok{=}\NormalTok{ rq.get(}\StringTok{"https://storage.googleapis.com/open.data.arnov.dev/static/plots/optimization/CostFunction2D.json"}\NormalTok{)}

\NormalTok{fig }\OperatorTok{=}\NormalTok{ go.FigureWidget(data}\OperatorTok{=}\NormalTok{data.json())}

\NormalTok{fig.update\_layout(}
    \OperatorTok{**}\NormalTok{template,}
\NormalTok{)}

\NormalTok{fig.show()}
\end{Highlighting}
\end{Shaded}

\begin{verbatim}
Unable to display output for mime type(s): text/html
\end{verbatim}

A 3D plot of the function
\(G(x,y) = \log\left(1 + \dfrac{A(x,y)^2}{\gamma + B(x,y)^2}\right)\)

\begin{verbatim}
Unable to display output for mime type(s): text/html
\end{verbatim}

\bookmarksetup{startatroot}

\hypertarget{introduction}{%
\chapter{Introduction}\label{introduction}}

This is a book created from markdown and executable code.

See Knuth (\protect\hyperlink{ref-knuth84}{1984}) for additional
discussion of literate programming.

\bookmarksetup{startatroot}

\hypertarget{summary}{%
\chapter{Summary}\label{summary}}

In summary, this book has no content whatsoever.

\bookmarksetup{startatroot}

\hypertarget{references}{%
\chapter*{References}\label{references}}
\addcontentsline{toc}{chapter}{References}

\markboth{References}{References}

\hypertarget{refs}{}
\begin{CSLReferences}{1}{0}
\leavevmode\vadjust pre{\hypertarget{ref-knuth84}{}}%
Knuth, Donald E. 1984. {``Literate Programming.''} \emph{Comput. J.} 27
(2): 97--111. \url{https://doi.org/10.1093/comjnl/27.2.97}.

\end{CSLReferences}

\hypertarget{latex-macros}{%
\section*{LaTeX Macros}\label{latex-macros}}
\addcontentsline{toc}{section}{LaTeX Macros}

\markright{LaTeX Macros}

\leavevmode\vadjust pre{\hypertarget{latex-macros}{}}%
\[\newcommand{\argmin}{\operatorname*{argmin}}\]
\[\newcommand{\argmax}{\operatorname*{argmax}}\]

\bookmarksetup{startatroot}

\hypertarget{about}{%
\chapter{About}\label{about}}

Here's a short introduction about me:

\hypertarget{what-i-do}{%
\section{What I do}\label{what-i-do}}

I'm currently working as a junior IT \& Data Consultant somewhere in the
\href{https://i.imgur.com/Bjj0AGm.png}{very pretty} business hub of
\href{https://goo.gl/maps/VMkGySJKihoE35At6}{La Défense}, but I'm also
trying to learn a bit more about programming and web development. I'm a
self-anointed pythonista 🐍, but I'm trying to learn Julia \& Javascript
aswell !

I also work a lot with \href{https://en.wikipedia.org/wiki/SQL}{SQL} at
my current job, though I'm eyeing on the recent trends like
\href{https://www.snowflake.com/}{Snowflake} and
\href{https://cloud.google.com/bigquery}{BigQuery} 👀.

Here's a picture of my workplace (and
\href{https://imgur.com/a/2MCtBRI}{some others I took in the area}):

\hypertarget{what-im-working-on}{%
\section{What I'm working on}\label{what-im-working-on}}

I'm currently working on a few projects, but I'm not sure I'll be able
to finish them anytime soon. I'm also trying to learn a bit more about
web development, so I'm trying to make a few websites. Here's a list of
the projects I'm currently working on:

\begin{longtable}[]{@{}
  >{\raggedright\arraybackslash}p{(\columnwidth - 4\tabcolsep) * \real{0.1000}}
  >{\raggedright\arraybackslash}p{(\columnwidth - 4\tabcolsep) * \real{0.1000}}
  >{\raggedright\arraybackslash}p{(\columnwidth - 4\tabcolsep) * \real{0.8000}}@{}}
\toprule()
\begin{minipage}[b]{\linewidth}\raggedright
Number
\end{minipage} & \begin{minipage}[b]{\linewidth}\raggedright
Language
\end{minipage} & \begin{minipage}[b]{\linewidth}\raggedright
Description
\end{minipage} \\
\midrule()
\endhead
1 & 🐍 & A few unpolished
\href{https://pypi.org/user/arnos-stuff/}{python packages} \\
2 & 🐍 & A python \href{https://api.arnov.dev/}{data API for eurostat
related data} \\
3 & JS & A \href{https://vue.arnov.dev/}{Vue.js website} to slowly learn
\href{https://vuejs.org/}{Vue.js} \\
4 & 🤓 & A small \href{https://write.as/arnov/}{blog about math related
topics} when I feel like talking theory \\
\bottomrule()
\end{longtable}

That's a lot ! But I also have some personnal matters \ldots{}

\bookmarksetup{startatroot}

\hypertarget{the-personal-stuff}{%
\chapter{The personal stuff}\label{the-personal-stuff}}

\hypertarget{im-autistic-queer}{%
\section{I'm Autistic \& Queer}\label{im-autistic-queer}}

I'm a proud
\href{https://www.health.harvard.edu/blog/what-is-neurodiversity-202111232645}{Aspie
/ Neurodivergent person}, I received an official diagnosis from
\href{https://www.hopital-saint-anne.fr/}{Hôpital Saint-Anne} in Paris,
France a year or so ago. They're considered a bit competent on the topic
so I trust them.

I'm also most probably trans 🏳️‍⚧️, but I'm not sure 100\% yet. I'm
currently trying to figure out how to go about it, the process is a bit
complicated in France, but it's a work in progress.

I wrote a short piece about the
\href{https://write.as/arnov/bayes-can-tell-you-youre-transgender}{relationship
between autism and gender diversity} a while back.

\hypertarget{i-like-games}{%
\section{I like games}\label{i-like-games}}

I'm also a big video game player, so maybe we can play together sometime
! I'm currently a bit busy with work, but I'd love to share some
factoids about my favorite games.

\hypertarget{i-like-music}{%
\section{I like music}\label{i-like-music}}

I'm a HUGE music nerd, just to name a few of my favorite genres:

\begin{longtable}[]{@{}
  >{\raggedright\arraybackslash}p{(\columnwidth - 2\tabcolsep) * \real{0.2000}}
  >{\raggedright\arraybackslash}p{(\columnwidth - 2\tabcolsep) * \real{0.8000}}@{}}
\toprule()
\begin{minipage}[b]{\linewidth}\raggedright
Genre
\end{minipage} & \begin{minipage}[b]{\linewidth}\raggedright
Description
\end{minipage} \\
\midrule()
\endhead
70's rock & especially progressive rock and Cantebury scene / Krautrock
bands \\
Bebop and Hard-Bop & The golden age of Bebop and Hard-Bop, especially
the work of Miles Davis, who helped me bridge the gap between Jazz and
Rock (I'm a huge fan of his work with John McLaughlin) \\
IDM & Especially the work of μ-Ziq, Floating Points, Squarepusher, and
Boards of Canada \\
Hip-Hop & The classics, especially the work of MF Doom, Madlib, and J
Dilla \\
Hyperpop & I'm a huge fan of the work of Charli XCX ! \\
K-pop & I've recently discovered I like K-pop thanks to League of
Legends' KDA group, and I'm slowly getting into the genre .. \\
Metal & I love Gojira obviously, but a lot of artists here and there,
the progressive Metal scene is pretty dope aswell 🤩 \\
\bottomrule()
\end{longtable}

\ldots{} and many more 🎵

\hypertarget{i-like-math}{%
\section{I like math}\label{i-like-math}}

I'm also a big math nerd, I'm currently trying to learn a bit more about
statistics and probability theory, as I'm a recovering tentative PhD
grad (I worked on
\href{https://en.wikipedia.org/wiki/Causal_inference}{Causal Inference}
back in
\href{https://en.wikipedia.org/wiki/Shanghai_Jiao_Tong_University}{Shanghai}).

I'd love learning more, but I'm busy and already have a lot to do !



\end{document}
